\documentclass[%
 reprint,
superscriptaddress,
%groupedaddress,
%unsortedaddress,
%runinaddress,
%frontmatterverbose, 
%preprint,
%preprintnumbers,
%nofootinbib,
%nobibnotes,
%bibnotes,
 amsmath,amssymb,
 aps,
%pra,
prb,
%rmp,
%prstab,
%prstper,
floatfix
]{revtex4-2}

\usepackage{graphicx}% Include figure files
\usepackage{dcolumn}% Align table columns on decimal point
\usepackage{bm}% bold math
%%%%%%%%%%%%%%%%%%%%%%%%%package by Zhong
\usepackage{units}
%\usepackage{lineno}
%\usepackage{float}
%\usepackage{amsmath}  % improve math presentation
%\usepackage{amsfonts}
%\usepackage{verbatim}
%%%%%%%%%%%%%%%%%%%%%%%%%%%%%%%%%%%%%%
\usepackage{hyperref}% add hypertext capabilities
\usepackage{xcolor}
\usepackage{xspace}

\newcommand{\bjoern}[2]{{\color{blue}{{\bf #1} #2}}}
\newcommand{\ahfx}{\ensuremath{\alpha_\text{HFX}}\xspace} 


\begin{document}


\title{Uncertainty Quantification in Multiscale Models of Charge Transport in Organic Semiconductors: Influence of the Exhange-Correlation Functional}

\author{Zhongquan Chen}
\author{Pim van der Hoorn}
\author{Bj\"orn Baumeier}
\affiliation{Department of Mathematics and Computer Science \& Institute for Complex Molecular Systems, Eindhoven University of Technology}

\date{\today}

\begin{abstract}
  \bjoern{TODO}{rewrite}This work quantifies uncertainties in a multiscale model for charge dynamics in organic semiconductors (OSCs). A first-principle multiscale model integrates classical molecular dynamics (MD), density functional theory (DFT), quantum mechanical calculation and continuous-time random walk (CTRW) processes to allow the simulation of charge transport in OSCs. The model brings inherent uncertainties arising from the empirical approximations and numerical simulations. Our work particularly focusing on quantifying the uncertainty in DFT exchange-correlation functional.

  By investigating the effects of varying Hartree-Fock (HF) levels within DFT functionals, we analyze the impact on electronic structures such as reorganization energy, molecule energy distribution, and coupling elements. The primary goal is to assess the robustness of predicting the quantity of interest (QoI), that is time-of-flight and charge mobility, under these uncertainties.
  
  Monte Carlo simulations and sensitivity analysis are employed to estimate the range and confidence levels of the predicted charge mobility. As shown by Sobol indices, the QoI is most sensitive to molecule energy, then reorganization energy and least sensitive to coupling elements.  
\end{abstract}

%\keywords{Suggested keywords}

\maketitle


\section{Introduction}
Organic semiconductors (OSCs) are materials that consist of organic molecules, often in disordered structures formed during spin-coating or deposition processing. Besides their semiconducting properties, they are mechanically flexible and come with the potential to control charge transport properties~\cite{hamers_flexible_2001,liu_high_2015,chow_organic_2020}. This flexible functionality is achieved by tuning molecular properties to achieve an ideal operation at the device level~\cite{bronstein_role_2020, bredas_organic_2002}. Computational approaches that aim at predicting the charge transport properties of disordered molecular materials resolving the interplay between single-molecule properties (such as their electronic structure or response properties) and mesoscale material morphology can play an important role in supporting or even guiding experimental optimization processes~\cite{bronstein_role_2020, sokolov_computational_2011, grynova_read_2018}.\bjoern{add citations}{}

\begin{figure*}[tbp]
  \centering
  \includegraphics[width=\linewidth]{figs/MSM2.pdf}
  \caption{\bjoern{TODO Bjoern}{Rewrite caption}The multiscale model workflow for OSC. Step 1: MD simulation to generate atom coordinates. Step 2: Molecule energy $E_i$ calculation. Step 3: Calculation of Reorganization energy $\lambda_{ij}$ and coupling element $J_{ij}$ for pairs of molecule $i,j$ whose COM distance is less than $r_\text{cutoff}$. Step 4: Modelling dynamics on device level, such as ToF calculation.}
  \label{fig:MSM}
\end{figure*}

First-principles based multiscale models are such types of computational approaches. Due to the infeasibility of explicitly simulating the coupled non-adiabatic electron-nuclear dynamics for the time- and length scales of realistic materials, these models typically exploit the localization of electronic states in disordered molecular materials and consider a rate-based description of hopping-type transport. A popular choice for the electron transfer rate $\omega_{ij}$ between two localization sites $i$ and $j$ is Marcus theory~\cite{marcus_theory_1956, marcus_electron_1993}  ~\bjoern{add}{citation}, in which
%
\begin{equation}
    \omega_{ij} = \frac{2\pi}{\hbar} \frac{|J_{ij}|^2}{\sqrt{4\pi \lambda_{ij} k_\text{B}T}} \exp\left(-\frac{(\Delta E_{ij} - \lambda)^2}{4\lambda k_\text{B}T}\right) ,
    \label{equ:Marcus}
\end{equation}
%
where $\hbar$ is the reduced Planck constant,  $k_\text{B}$ the Boltzmann constant, and $T$ the temperature. As depicted in Fig.~\ref{fig:MSM}, given a larger scale morphology and with the definition of localization sites (Fig.~\ref{fig:MSM}(a)), multiscale models employ first-principles based methods (Fig.~\ref{fig:MSM}(b)) to explicitly calculate the remaining physical, material-specific (or rather transfer-pair-specific) quantities in Eq.~\ref{equ:Marcus}: the reorganization energy $\lambda$ (here a single value as we assume a single-component material), the electronic coupling $J_{ij}$, and the site energy difference $\Delta E_{ij} = E_i - E_j$. With all that information, charge transport is modeled as a continuous time random walk (CTRW) process on a graph $\mathbf{G}$, constructed from the localization sites and the calculated rates between them (Fig.~\ref{fig:MSM}(c)). Panel (b) of Fig.~\ref{fig:MSM} mentions some specific methods for the calculation of the quantities entering Eq.~\ref{equ:Marcus}, such as mixed quantum-classical methods for obtaining the site energies with the help of microelectrostatic methods~\cite{poelking_impact_2015, poelking_long-range_2016} \bjoern{add}{citations} or the dimer projection method for determining coupling elements~\cite{baumeier_density_2010}, as they are the ones adopted in this work. What is essential about these and alternative ones is that they typically rely on density-functional theory (DFT) calculations, either directly or as a mean to parametrize classical models. The dependence of DFT calculations on the choice of an exchange-correlation functional raises the question of how sensitive the simulated charge transport is to this choice and how certain predictions of material properties are. 

Uncertainty quantification (UQ) is concept from computational science which allows for an estimation of confidence intervals for a quantity of interest (QoI) and an analysis of its sensitivity in models of a physical system that contain uncertain, maybe empirical, or noisy, parameters~\cite{sarkar_uncertainty_2017,oconnor_quantifying_2024, chernatynskiy_uncertainty_2013, suleimenova_tutorial_2021,coveney_reliability_2021, coveney_when_2021}. Many common UQ studies focus on models with (partial) differential equations, e.g., drift-diffusion equations in which the diffusivity as parameter, and assume a certain distribution for the values of the parameter(s). For the multiscale model of charge transport, it is not straightforward to cast the large variety of available exchange-correlation functionals into the role of a model parameter with some distribution. To keep the problem tractable, we focus here instead specifically on the exchange part in hybrid functionals~\cite{perdew_rationale_1996,marques_densitybased_2011}, in which a DFT model for the exchange is mixed using a weighting factor $\alpha_\text{HFX}$ with a Hartree--Fock type exchange, i.e.,
%
\begin{equation}
  E_\text{x} = \alpha_\text{HFX} E_\text{x}^\text{HF} + (1-\alpha_\text{HFX})E_\text{x}^\text{DFT}.
  \label{equ:hybrid}
\end{equation}
%
More specifically, we take as the basis the PBE0 functional~\cite{adamo_toward_1999} and scrutinize (i) how the predictions of the multiscale model of charge transport are affected by variation of $\alpha_\text{HFX}$ as a proxy for uncertainties in the choice of DFT functionals, (ii) what the level of confidence is in quantitative predictions, and (iii) what are the most sensitive quantities in the model. In this sense, Eq.~\ref{equ:hybrid} is deceptively simple. For each value of $\alpha_\text{HFX}$, the graph $G$ is constructed using the respective value of the reorganization energy, the $N_\text{mol}$ site energies, and the $N_\text{pair}$ coupling elements, and the dimensionality of the problem from the perspective of UQ is $N_\text{UQ}=1+N_\text{mol}+N_\text{pair}$, which can easily be on the order of $10^{4}-10^{5}$. We consider the simulated time-of-flight (ToF) and the associated mobility the QoI in the following (see Fig.~\ref{fig:MSM}(c)) which then are subject to uncertainties in these $N_\text{UQ}$ parameters, {\em stemming} from the variation in $\alpha_\text{HFX}$. As a prototypical system, we will study hole transport in an amorphous morphology of 2-methyl-9, 10-bis(naphthalen-2-yl)anthracene (MADN), a wide-gap semiconductor that is used extensively as an ambipolar host material in organic light-emitting diodes~\cite{ko_accurate_2019, chang_great_2017} ~\bjoern{add}{citation}. 

This paper is organized as follows. Section~\ref{sec:model} outlines the theoretical and computational details of the multiscale model, including the morphology simulation with classical molecular dynamics, the determination of all quantities in the transition rates, and the calculation of the charge transport properties from the constructed graph model. In Section~\ref{sec:MSMresults}, we discuss the explicit results from the model using different values of $\alpha_\text{HFX}$, before we show the results of uncertainty quantification and sensitivity analysis via Monte Carlo sampling. A brief summary concludes the paper. 

\section{Multiscale Model}
\label{sec:model}
The multiscale model maps a large scale molecular morphology with atomistic detail into a graph $\mathbf{G}(\mathbf{V}, \mathbf{W})$, where the set of nodes $\mathbf{V}$ is determined from the center-of-masses of the individual molecules and $\mathbf{W}$ is the adjacency matrix formed by the Marcus rates $\omega_{ij}$. Two nodes $i,j$ are connected if the corresponding molecules have their closest-contact distance smaller than $r_\text{cutoff}=\unit[0.5]{nm}$. 

\subsection{Molecular Dynamics}
Classical molecular dynamics (MD) is used to create an amorphous morphology of MADN. An empirical force-field for these simulations has been obtained via the Automated Topology Builder~\cite{stroet_automated_2018}, and an initial structure containing 1000 molecules in a cubic cell is created. Periodic boundary conditions are applied throughout in all three spatial directions. 
After energy minimization, the system is simulated for \unit[1]{ns} in the $NpT$ ensemble, keeping a constant temperature of \unit[300]{K} and constant pressure of \unit[1]{bar} using the velocity-rescale thermostat~\cite{bussi_canonical_2007} with the coupling time constant \unit[0.1]{ps} and the Parrinello-Rahman barostat~\cite{parrinello_polymorphic_1981} with a time constant for pressure coupling \unit[2]{ps}. The equation of motion for updating the atomic coordinates is implemented by leap-frog algorithm~\cite{van_gunsteren_leap} with a time step of \unit[1]{fs}. Following this, the temperature is increased to \unit[800]{K}, well above the glass transition temperature, during a period of \unit[0.5]{ns}. This temperature is maintained for \unit[1]{ns} before cooling back down to \unit[300]{K} during a period of \unit[0.5]{ns}. Such a heating-cooling cycle is repeated three times. After this simulated annealing, a production run is conducted for \unit[2]{ns} using the $NpT$ ensemble. The final configuration of MADN is chosen for the further steps in the multiscale model, whose configuration is a cubic box with a length of \unit[9.0]{nm} and a density of $\unit[1.08]{g/cm^3}$. All calculations have been performed with the GROMACS software package~\cite{berendsen_gromacs_1995}.

\subsection{Electronic Structure Calculations} 
\label{sec:es}
Molecular orbitals $\phi_l (\mathbf{r})$ with energies $\epsilon_l$ of the individual molecules in the morphology are obtained within DFT as the solutions to the  Kohn--Sham equations~\cite{kohn_self_1965}
%
\begin{eqnarray}
    && \left(-\frac{1}{2}\nabla^2_{\mathbf{r}} + v_\text{ext}(\mathbf{r}) + v_\text{H}[\rho](\mathbf{r}) + v_\text{XC}[\rho](\mathbf{r})\right) \phi_l(\mathbf{r}) \nonumber \\
    && = H^\text{KS} \phi_l(\mathbf{r}) = \epsilon_l \phi_l (\mathbf{r}) ,
    \label{eq:KS2}
\end{eqnarray}
%
where $v_\text{ext}$ in an external potential (typically from the nuclei), $v_\text{H}[\rho]$ the electrostatic Hartree potential of a classical charge density $\rho(\mathbf{r})$, and $v_\text{XC}[\rho]$ the exchange-correlation potential containing explicit quantum-mechanical electron-electron interactions. The charge density is determined from the single-particle wave functions as $\rho(\mathbf{r})=\sum\limits_{l=1}^{N_\text{el}} \left\vert\phi_l(\mathbf{r})\right\vert^2$. As the Hartree and exchange-correlation potential depend on this density, solutions to Eq.~\ref{eq:KS2} have to be found self-consistently. This corresponds to finding the ground-state density $\rho_0$ that minimizes the total energy of the system expressed as
%
\begin{equation}
    U=U[\rho] = T_s[\rho] + \int v_\text{ext}(\mathbf{r}) \rho(\mathbf{r}) d \vec{r} + E_\text{H}[\rho] + E_\text{XC}[\rho]
    \label{eq:KS_model}
\end{equation}
%
where $T_s[\rho]$ is the kinetic energy, $E_\text{H}[\rho]$ and $E_\text{XC}[\rho]$ the Hartree and exchange-correlation energies, respectively. 

The practical calculations in this work have been performed with the ORCA software~\cite{Neese2012a} using the def2-tzvp~\cite{weigend_accurate_2006} basis set to represent the Kohn--Sham wave functions. As mentioned in the Introduction, the correlation part of the exchange-correlation functional is taken from the PBE0 functional, while the weighting factor of the Hartree--Fock type exchange in the exchange part is varied. In common practice, $\alpha$ is small and below 0.25. For our uncertain quantification study, $\alpha=0,0.05,0.10,0.15,0.20,0.25$ are chosen for the multiscale model.

\subsection{Reorganiztion Energy}
 The reorganization energy $\lambda$ accounts for the energy change caused by the geometry variation during the charge transport, and is linked to four points on the potential energy surfaces of neutral (n) and charged (c) molecules at neutral (N) or charged (C) equilibrium geometries via:
%
\begin{equation}
    \lambda_{ij} = U_i^\text{nC} - U_i^\text{nN} + U_j^\text{cN} - U_j^\text{cC},
    \label{eq:lambda}
\end{equation}
%
where $U^\text{xX}$ is the total DFT energy of $\text{x}=\text{n},\text{c}$ state in the $\text{X}=\text{N},\text{C}$ geometry. While in principle transfer pair specific, we use a single value of all molecular pairs.

\subsection{Site Energy}
The site energy $E_i = E_i^\text{c} - E_i^\text{n}$ is the difference between the total energies of the system in which molecule $i$ is carrying a charge or not, corresponding to the ionization potential in case of hole transport and the negative of the electron affinity in case of electron transport. The individual total energies in turn consist of different contributions associated with different physical mechanism, i.e.,
%
\begin{equation}
E_i^x = U_i^\text{xX} + E_i^{\text{x},\text{el}} + E_i^{\text{x},\text{polar}},
\label{eq:Es}
\end{equation}
%
where $U_i^\text{xX}$ is the internal energy contributions and both $E_i^{\text{x},\text{el}}$ and $E_i^{\text{x},\text{polar}}$ are contributions arising from purely static and polarizable intermolecular interactions, respectively. As those interactions are typically long-ranged, the intermolecular contributions to the site energy can typically not be calculated with a fully quantum-mechanical method and classical models are adopted, instead, which we refer to as a microelectrostatic model using moment representations parametrized based on single molecule DFT reference data. Specifically, in the multiscale model here, we employ a point charge respresentation~\cite{jcc540110311} for the electrostatic potential of charged and neutral molecules, so that the electrostatic energy contribution is
%
\begin{equation}
    E_i^{\text{x},\text{el}} = \frac{1}{4 \pi \epsilon_0} \sum\limits_{a_i} \sum\limits_{b_k,k \neq i} \frac{q^\text{x}_{a_i}q^n_{b_k}}{ |\mathbf{R}_{a_i} - \mathbf{R}_{b_k}|} 
\end{equation}
%
where $\epsilon_0$ is the vacuum permittivity, $a_i, b_k$ denotes the atoms in molecule $i,k$, $q^{\text{x}}_{a_i}$ are the partial charge of atom $a$ when molecule $i$ is in state $\text{x}$. To account for effects of polarization, and to evaluate $ E_i^{\text{x},\text{polar}}$, we use the model of distributed atomic dipole polarizabilities (Thole model)~\cite{thole_molecular_1981}, in which the parameters are also determined such that the classical volume of the molecular polarizability tensor matches the DFT reference.
Intermoelcular effects are considered in a region of \unit[4.0]{nm} around each individual molecule. Practical calculations of the site energies are performed using the VOTCA software~\cite{Baumeier2011,doi:10.1021/acs.jctc.8b00617,10.1063/1.5144277,Baumeier2024}.

\subsection{Electronic Coupling Elements}
The coupling element $J_{ij}$ between molecule $i$ and $j$ describe the coupling strength between two localized states, here approximated by monomer single-particle wavefunctions $\vert \phi_i\rangle$ and $\vert\phi_j\rangle$, respectively. For hole transport, the relevant orbitals are highest-occupied molecular orbital (HOMO). Using the Dimer-Projection Method~\cite{baumeier_density_2010} the coupling element is determined as:
%
\begin{equation}
    J_{ij} = \frac{ J^0_{ij}- \frac{1}{2}(e_i+e_j) S_{ij} }{ 1- S_{ij}^2 }
    \label{equ:JAB}
\end{equation}
%
where $J^0_{ij} = \langle \phi_i | \hat{H}^\text{KS}_\text{D} | \phi_j \rangle $, $e_i = \langle \phi_i | \hat{H}^\text{KS}_\text{D} | \phi_i \rangle $, $e_j = \langle \phi_j | \hat{H}^\text{KS}_\text{D} | \phi_j \rangle $, and $S_{ij}=\langle \phi_i | \phi_j \rangle $ with bra-ket notation. The Hamiltonian of the dimer,  $H^\text{KS}_\text{D}$ (see Eq.~\ref{eq:KS2}), is diagonal in its eigenbasis $\left\{\vert \phi^\text{D}_k\rangle\right\}$ with eigenvalues $\left\{ \epsilon^\text{D}_k\right\}$, so $H^\text{KS}_\text{D} = \text{diag}(\epsilon^\text{D})$. With the projections of the monomer functions on the dimer eigenbasis, i.e., $p_{ik} = \langle \phi_i | \phi^\text{D}_k \rangle$ and  $p_{jk} = \langle \phi_j | \phi^\text{D}_k \rangle$, $J^0_{ij}$ can be calculated as $J^0_{ij} = \mathbf{p}_i^\text{T} \text{diag}(\epsilon^\text{D}) \mathbf{p}_j$. Similarly, $e_{i(j)} = \mathbf{p}_{i(j)}^\text{T} \mathbf{p}_{i(j)}$ and $S_{ij} =  \mathbf{p}_i^\text{T} \mathbf{p}_j$. All  of these operations are performed in the basis set representation of the Kohn--Sham wave functions (see Sec.~\ref{sec:es}) as implemented in VOTCA.

\subsection{Time-of-flight Calculation}
After constructing the  graph $\mathbf{G}$ from the multiscale model, charge dynamics are modeled as a continuous-time random walk on this graph. In the time-of-flight (ToF) model, some vertices serve as source nodes, representing the electrode where charge carriers are injected, and some as sink nodes, where charge carriers are detected and the ToF is recorded. In CTRW the ToF is calculated as the expected hitting time of a continuous time Markov chain. For a system with $N$ molecules and one charge carrier, the transition rates between the molecules define the adjacency matrix $\mathbf{W}$:
%
\begin{equation}
  \label{eq:transition_rates}
	\mathbf{W}_{ij} =
	\begin{cases}
	     0			&  i \text{ is not connected to } j,\\
         \omega_{ij}   &  i \text{ is connected to } j,
	\end{cases}
\end{equation}

Then the transition probability from state $i$ to $j$ is $p_{ij} = \mathbf{W}_{ij}/D_i$ where $D_i := \sum_{j} \mathbf{W}_{ij}$.
And the expected time from node $i$ to reach the sink $\tau_i$ is calculated via: 
%
\begin{equation}
  \label{eq:hitting_time}
	\tau_i = \begin{cases}
		\frac{1}{D_i} + \sum_{j \ne i} p_{ij} \tau_{j} &\text{if $i$ is not a sink node},\\
		0 &\text{else.} 
	\end{cases}
\end{equation} 

To account for all possible starting nodes of the carrier, all source nodes must be considered. The random walk process can be modeled as a parallel electric network of capacitors~\cite{doyle_random_1984}. Accordingly, the ToF is evaluated using the harmonic mean:
%
\begin{equation} 
  \tau = N_\text{source} \left[\sum_{i \in \text{Source}} (\tau_i)^{-1}\right]^{-1},
  \label{eq:ToF}
\end{equation}
%
where $N_\text{source}$ is the number of source nodes.

To determine the ToF in the simulated MADN system along the positive $x$-direction, we remove the PBC in this direction and define the nodes with coordinates $\unit[0]{nm} < x < \unit[0.5]{nm}$ as source nodes, and those with coordinates $\unit[8.5]{nm} < x < \unit[9]{nm}$ as sink nodes. These definitions are changed accordingly for simulating transport in negative $x$-direction, and $y$- or $z$-directions.

\section{Explicit Results from the Multiscale Model}
\label{sec:MSMresults}
In this section, we present and analyze the explicit results of the multiscale model of charge transport in amorphous MADN as obtained for different values of $\alpha_\text{HFX}$.



\subsection{Molecular Parameters}

\begin{figure}[tbp]
    \centering
    \includegraphics[width=\linewidth]{figs/fig_autogen.pdf}
    \caption{Dependence of molecular parameters as used directly in the multiscale model or in its paramaterization phase on the amount of Hartree--Fock type exchange in the PBE0-based hybrid functional $\alpha_\text{HFX}$. (a) The adiabatic ionization potential, (b) reorganization energy, (c) dipole moment of the neutral molecule, and (d) isotropic molecular polarizability in neutral and charged states, respectively.}
    \label{fig:autogen_MADN}
\end{figure}

We begin with a brief discussion of the molecular parameters as they are used in different ways in the multiscale model. Figure~\ref{fig:autogen_MADN} shows the adiabatic ionization potential, reorganization energy, neutral state dipole moment, and isotropic polarizability in neutral and cationic (hole) states, respectively. The adiabatic ionization potential in Fig.~\ref{fig:autogen_MADN}(a) in principle contributes to the site energy but as it is determined per molecule-type in the system, it has no effect on the site-energy difference $\Delta E_{ij}$ in Eq.~\ref{equ:Marcus}. It is nevertheless interesting to see that it increases almost linearly over the shown range of \ahfx. In contrast, the reorganization energy as shown in Fig.~\ref{fig:autogen_MADN}(b) appears to saturate for \ahfx after an initially close to linear increase. In total, $\lambda$ is found to be in an interval between \unit[0.246]{eV} and \unit[0.316]{eV} \bjoern{need}{correct numbers here}. Panels (c) and (d) of Fig.~\ref{fig:autogen_MADN} show the dipole moment of the neutral MADN molecule and the isotropic polarizability of the neutral and cationic (hole) states, respectively, both as electrostatic properties that enter indirectly the parameterization of the microelectrostatic model. As is visible, the dipole moment is rather independent on \ahfx (note that the jump of the last shown data point appears more pronounced because of the vert small scale on the $y$-axis). The isotropic polarizabilities in panel (d) exhibit a linear decrease with increasing value of \ahfx which can be attributed to an increasingly attractive effective potential from stronger the Hartree--Fock-like exchange term and consequently more strongly bound electrons as one can also see from the increaseing ionization potential in panel (a).  

\subsection{Distributions of site energies}

\bjoern{STOP}{HERE}


%\begin{table*}[tbp]%The best place to locate the table environment is directly after its first reference in text
%\caption{\label{tab:W2_E}%
%The 1-Wasserstein distance between the distributions pair of molecule energy $P_{E(\alpha)}$, electrostatic energy $P_{E^\text{el}(\alpha)}$, polar energy $P_{E^\text{polar}(\alpha)}$ and $P_{\log_{10}(J_{ij}^2)(\alpha)}$.
%}
%\begin{ruledtabular}
%  \begin{tabular}{c c c c c c}
%  $\alpha'$ & 0 & 0.05 & 0.10 & 0.15 & 0.20 \\
%    \hline
%  $W_1 (P_{E(\alpha')}, P_{E(\alpha=0.25)})$ &  0.0054 & 0.0091 & 0.0067 & 0.0053 & 0.0045 \\
%  $W_1 (P_{E^\text{el}(\alpha')}, P_{E^\text{el}(\alpha=0.25)})$ &  0.010 & 0.012 & 0.0089 & 0.0070 & 0.0032 \\
%  $W_1 (P_{E^\text{polar}(\alpha')}, P_{E^\text{polar}(\alpha=0.25)})$ &  0.0065 & 0.0043 & 0.0027 & 0.0023 & 0.0031 \\
%  $W_1 (P_{\log_{10}(J_{ij}^2)(\alpha')}, P_{\log_{10}(J_{ij}^2)(\alpha=0.25)})$ &  0.0054 & 0.0091 & 0.0067 & 0.0053 & 0.0045 \\
%    \end{tabular}
%\end{ruledtabular}
%\end{table*}
%
  
\begin{figure*}
  \centering
  \includegraphics[width=0.80\textwidth]{figs/scatterE_qmmm.pdf}
  \caption{\bjoern{TODO Zhong:}{add labels (a)...; replace $\alpha$ with $\alpha_\text{HF}$.}Scatter plot of site energy of MADN molecules calculated from different HFX, compared to the site energy calculated from HFX=0.25 (The PBE0 functional). The brighter color near the diagonal lines indicates denser population of the molecules.  The top and right histogram show the energy distributions.}
  \label{fig:E_qmmm_MADN}
\end{figure*}



Figure \ref{fig:E_qmmm_MADN} presents a scatter plot comparing the site energies of MADN molecules across different HFX values, using $\alpha$ = 0.25 (PBE0 functional) as a reference for which the calculated molecule energy has a range from -1.17 to -0.60 \unit[]{eV}y.
The range of molecule energy for the other five HFX are: 
from -1.11 to -0.59 \unit[]{eV} for $\alpha=0$, 
from -1.07 to -0.63 \unit[]{eV} for $\alpha=0.05$, 
from -1.13 to -0.61 \unit[]{eV} for $\alpha=0.10$
from -1.16 to -0.60 \unit[]{eV} for $\alpha=0.15$, 
and from -1.12 to -0.63 \unit[]{eV} for $\alpha=0.20$.
The bright-yellow color in all plots of Fig. \ref{fig:E_qmmm_MADN} indicates that most data points are clustering near the diagonal line, so the molecular energies calculated by different HFX are very close.

%To quantify the similarity between two empirical distribution, Wasserstein distance~\cite{villani_optimal_2009} is used.
%Denote the empirical distribution of molecular energy $E$ at $\alpha=0$ to be $P_{E(\alpha=0)}$. 
%The 1-Wasserstein distance between $P_{E(\alpha=0)}$ and $P_{E(\alpha=0.25)}$ is calculated as:
%\begin{eqnarray}
%    && W_1 (P_{E(\alpha=0)}, P_{E(\alpha=0.25)}) = \nonumber \\
%    && \frac{1}{N} \sum\limits_{i=1}^N | E_{(i)}(\alpha=0) - E_{(i)}(\alpha=0.25) |
%\end{eqnarray}
%where $E_{(i)}(\alpha=0)$ is the order statistics of $E_i(\alpha=0)$. The 1-Wasserstein distance between two empirical distributions for each plot in Fig. \ref{fig:E_qmmm_MADN} is shown in Table \ref{tab:W2_E}. $W_1 (P_{E(\alpha=0)}, P_{E(\alpha=0.25)})$ has range between 0.0045 to 0.0091, indicating that the molecular energy distributions are very similar. 


To further estimate the energy variation due to HFX, one can look at the molecule that has maximum $|E_i(\alpha=0.25) - E_i(\alpha')|$ for all molecule index $i=1,2,\cdots,N$. 
This quantity provides information of the amount of the largest variation one can expect in $E_i(\alpha)$. Index the molecule that has maximum $|E_i(\alpha=0.25) - E_i(\alpha')|$ to be $i'$. 
If molecule $i'$ energy is the lowest one among the $N$ molecules, then a further change (such as lowering) in the energy of molecule $i'$ may lead to large change in ToF, since low energy potentially lead to a trap in the charge transport dynamics and significantly affects the mobility. 

Table \ref{tab:maxEi} shows that depending on $\alpha'$, the maximum $|E_i(\alpha=0.25) - E_i(\alpha')|$ is 0.12 to 0.15, which is considerable small compared to the range of molecular energy and the magnitude of the energy disorder, which is around 0.08 as shown in Fig. \ref{fig:autogen_MADN}. 
The energy of molecule $i'$ shown in Table \ref{tab:maxEi} indicates that the $i'$ molecule energy is not the lowest one in the system. So the maximum energy variation quantified by $\max(|E_i(\alpha=0.25) - E_i(\alpha')|)$ is not likely to results in large change in ToF due to trapping effects. 

\begin{table}[tbp]
\caption{\label{tab:maxEi}%
In the tested $\alpha'$ values, the second column shows the maximum of $E_i(\alpha=0.25) - E_i(\alpha')$ for all molecules. Denoting $i'$ as the index of the molecule which has maximum $E_i(\alpha=0.25) - E_i(\alpha')$ for the chosen $\alpha'$, the third column shows the energy of molecule $i'$ calculated by $\alpha=0.25$. The fourth column shows the molecule energy $E_{i'}(\alpha')$ calculated by $\alpha'$. All the energies are in unit $\unit[]{eV}$.
}
\begin{ruledtabular}
  \begin{tabular}{c c c c}
  $\alpha'$ & $\max(|E_i(\alpha=0.25) - E_i(\alpha')|)$  & $E_{i'}(\alpha=0.25)$ & $E_{i'}(\alpha')$ \\
    \hline
  0 & 0.13 &  -1.07 & -0.94 \\
  0.05 & 0.12 & -0.82 & -0.94 \\
  0.10 & 0.13 & -1.07 & -0.94 \\
  0.15 & 0.15 & -0.84 & -0.69 \\
  0.20 & 0.12 & -0.70 & -0.82 \\
    \end{tabular}
\end{ruledtabular}
\end{table}

Figures \ref{fig:Estat_qmmm_MADN} and \ref{fig:Edip_qmmm_MADN} show that both electrostatic and polarization energies remain consistently close to the ones calculated from $\alpha=0.25$, since all the points comparing $E^\text{el}_i(\alpha')$ and $E^\text{el}_i(\alpha=0.25)$ in the scatter plot are close to the diagonal line.
The range of $E^\text{el}$ calculated by different $\alpha$ is as follows:
from  -0.55 to 0.078 eV for $\alpha=0$, 
from  -0.52 to 0.047 eV for $\alpha=0.05$, 
from  -0.57 to 0.086 eV for $\alpha=0.10$, 
from  -0.59 to 0.14 eV for $\alpha=0.15$, 
from  -0.57 to 0.10 eV for $\alpha=0.20$, 
and from -0.63 to 0.10 eV for $\alpha=0.25$.


The 1-Wasserstein distance between the distribution of $E^\text{el}$ calculated by $\alpha'$ and that calculated by $\alpha=0.25$, $W_1(P_{E^\text{el}(\alpha')}, P_{E^\text{el}(\alpha=0.25)})$, is shown at the third row at Table \ref{tab:W2_E}. 
The maximum of $W_1(P_{E^\text{el}(\alpha')}, P_{E^\text{el}(\alpha=0.25)})$ 0.012. The 1-Wasserstein distance between the $E^\text{polar}$ distribution calculated by $\alpha'$ and that calculated by $\alpha=0.25$ is also shown in Table \ref{tab:W2_E}, with a maximum of 0.006. So the small Wasserstein distances suggest that the using different $\alpha'$ other than $\alpha=0.25$, the data of $E, E^\text{el}, E^\text{polar}$ have not changed significantly.

The standard deviations of the electrostatic energy are 0.099, 0.094, 0.099, 0.11, 0.10, \unit[0.10]{eV}, and the standard deviations of the polarization energies are 0.044, 0.043, 0.043, 0.047, 0.045, \unit[0.045]{eV}. So the disorder in electrostatic energies is consistently larger than that in molecule energies, and the palorization effect reduces the electrostatic energy disorder. 
Each individual molecule has similar polarization energies under different HFX calculation, in spite of $P_\text{iso}$ is linearly dependent on HFX. 



\subsection{Distributions of electronic couplings}


\begin{figure*}
  \centering
  \includegraphics[width=0.80\textwidth]{figs/scatterJ_all.pdf}
  \caption{\bjoern{TODO Zhong:}{add labels (a)...; replace $\alpha$ with $\alpha_\text{HF}$; it should be $\log_{10}(J^2/(\unit[]{eV})^2)$; make sure the 10 of $\log_{10}$ is not touching $J$.} Scatter plot of MADN coupling element $\log_{10} J^2$ calculated from different HFX, compared to the polarization energy calculated from HFX=0.25 (The PBE0 functional). The brighter color near the diagonal lines indicates denser population of the molecules.  The top and right histogram show the energy distributions.}
  \label{fig:J_MADN}
\end{figure*}

The coupling element obtained with different HFX for each molecular pair compared to that obtained with HFX=0.25 is shown in the scatter plot Fig. \ref{fig:J_MADN}.


The distribution of the majority $\log_{10} J_{ij}^2$ has a range of -14 to -2 while only a few $\log_{10} J_{ij}^2$ has low value below -15. For all HFX values, the distribution has a peak at around -5. 
For large values $\log_{10} J_{ij}^2 < -5$ the coupling element calculated from $\alpha'=0, 0.05, 0.10, 0.15, 0.2$ remains consistently close to that calculated by $\alpha=0.25$. While for small values $\log_{10} J_{ij}^2 > -5$, the $\alpha'$ calculated coupling elements for a specific molecular pair can have large variation compared to $\log_{10} J_{ij}^2$ calculated using $\alpha=0.25$.
The 1-Wasserstein distance between the $J$ distributions obtained by different $\alpha$ is always smaller than 0.01,  
indicating similar $J$ distributions even though a few $J_{ij}$ shows large difference when $\alpha$ varies. 

In charge transport network, $J_{ij}^2$ is undirectional for a molecular pair, and decays exponentially as the mutual center-of-mass distance increases. The large $\log_{10} J_{ij}^2$ are usually observed when molecular $i$ and $j$ are close.
When the center-of-mass distance is very large, the $\log_{10} J_{ij}^2$ is small and some of those small $\log_{10} J_{ij}^2$ are sensitive to $\alpha$. 

However, the charge dynamics is dominated by large $\log_{10} J_{ij}^2$ which are found between molecule pair having small center-of-mass. As the center-of-mass distance increases, $\log_{10} J^2$ becomes small and negligible for the charge dynamics. So the small $J_{ij}$ may not have noticeable effect on the ToF although their calculated values from different HFX are very different. 

To determine the range of $\log_{10} J^2$ that are significant for the charge dynamics, a percolation algorithm is performed. This algorithm aims to find the largest size of subgraphs after removing the edges whose coupling elements are below a critical value. The procedure is as follows:
\begin{enumerate}
    \item A critical value $J_c$ is chosen,
    \item In graph $\mathbf{G}$ remove all the edges with $J_{ij}^2 < J_c^2$
    \item Calculate the maximum size $\max({N_\text{sub}})$ of the connected subgraphs
\end{enumerate}

\begin{figure}[btp]
  \centering
  \includegraphics[width=0.45\textwidth]{figs/fig_network_all.pdf}
  \caption{\bjoern{TODO Zhong:}{same comments regarding the $\log_{10}$ as above; replace $\alpha$ with $\alpha_\text{HF}$.}The maximum size of the subgraphs obtained in the percolation algorithm as a function of critical value $J_c$. The molecular systems obtained by different HFX are colored according to the legend.}
  \label{fig:J_percolate}
\end{figure}

\begin{table}[t]
  \caption{\label{tab:ToF_J} The ToF along a distance $L_x = 9 \unit[]{nm}$  of the MADN system as a function of the HFX. The molecular connections with coupling element $\log_{10} J^2_{ij} < 5$ are deleted.}
  \centering
  \begin{tabular}{c c c c }
  \hline
  \hline
      $\alpha$ & & & ToF [s]  \\
  \hline
      0.00 & & &  $7.5 \times 10^{-9}$ \\
      0.05 & & & $ 8.1 \times 10^{-9}$ \\
      0.10 & & & $ 1.6 \times 10^{-8}$ \\
      0.15 & & & $ 3.1 \times 10^{-8}$ \\
      0.20 & & & $ 2.5 \times 10^{-8}$ \\
      0.25 & & & $ 9.7 \times 10^{-8}$ \\
  \hline
  \hline
  \end{tabular}
\end{table}

After performing percolation algorithm, Figure \ref{fig:J_percolate} shows the dependence of $\max({N_\text{sub}})$ on the critical value $J_c$. When $\log_{10} J_c^2 < 5$, $\max({N_\text{sub}})$ is 1000, meaning that removing all the edges with $J_{ij}^2 < 1 \times 10^{-5}$ the 1000 vertexes still form a connected graph. 
But when removing the edges with $J_{ij}^2 < 3.1 \times 10^{-4}$ ( corresponding to $J_c > -3.5$), the maximum size subgraph is 600, and it quickly decreases as $J_c$ increases, showing a phase transition. 
So the dominated coupling elements are those with $\log_{10} J_c^2 > -5$. 
And the variation in small $J_{ij}$ due to HFX has less impact on the charge dynamics. Table \ref{tab:ToF_J} shows the ToF for the MADN system after deleting the molecular connections with  $\log_{10} J_c^2 < -5$. The resulting ToF is close to the values obtained without deleting any connections (shown in Table \ref{tab:ToF_MADN_HFX}).
The increase of ToF after deleting the connections with  $\log_{10} J_c^2 < -5$ is always less than 30\%, confirming that the molecular connections with the small $J_{ij}$ has small effect in ToF. 

In this section we answer the question of how does HF level change the electronic structures ($\lambda, E_i, J_{ij}$) and 
how does HF level change the ToF. 
In the next section, uncertainty quantification via Monte Carlo method and sensitivity analysis is performed to answer which electronic structure (among energy, coupling element and reorganization energy) uncertainty has the most impact on the ToF distribution. 
And further we estimate the range of the quantity of interest, given a confidence level.


The ToF is also calculated without energy disorder, meaning $\Delta E_{ij}=0$ for all $i,j$.
Table~\ref{tab:ToF_MADN_HFX} shows the ToF and corresponding diffusive velocity $v_\text{ToF}$.
With energy disorder, the ToF increases by a factor of approximately 15 as $\alpha$ increases from 0 to 0.25.  
While without energy disorder, the increase factor is about 4. This increase in ToF is attributed to the rise in $\lambda$, changes in molecular energies, and variations in coupling elements.

\begin{table}[tbp]
\caption{The ToF and diffusive velocity calculated by ToF along a distance $L_x = \unit[9.0]{nm}$ with and without energy disorder of the MADN system as a function of the HFX. The ToF is in unit $\unit[]{s}$ and $v_\text{ToF}$ in unit $\unit[]{m/s}$.}
\begin{ruledtabular}
  \begin{tabular}{c c c c c}
        $\alpha$ & ToF& $v_\text{ToF}$ & ToF(no $E$)& $v_\text{ToF}$(no $E$) \\
    \hline
        0.00 &  $6.4 \times 10^{-9}$ & 1.2 & $1.9 \times 10^{-10}$ & 42 \\
        0.05 & $ 7.9 \times 10^{-9}$ & 1.0 & $4.1 \times 10^{-10}$ & 20 \\
        0.10 & $ 1.6 \times 10^{-8}$ & 0.51 & $4.0 \times 10^{-10} $ & 20 \\
        0.15 & $ 3.0 \times 10^{-8}$ & 0.31 & $4.0 \times 10^{-10} $ & 25 \\
        0.20 & $ 2.1 \times 10^{-8}$ & 0.39 & $4.5 \times 10^{-10}$ & 18 \\
        0.25 & $ 9.5 \times 10^{-8}$ & 0.083 & $7.2 \times 10^{-10}$ & 11 \\
    \end{tabular}
\end{ruledtabular}
\label{tab:ToF_MADN_HFX}
\end{table}


%%%%%%%%%%%%%%%%%%%%%%%%%%%%%%%%%%%%%%%%%%%%%%%%%%%%%%%%%

\section{Uncertainty Quantification and Sensitivity Analysis}
\label{sec:UQ}
The previous section shows that different HFX affect the calculated ToF. In this section, we use the Monte Carlo method to estimate the range of the ToF given a confidence level, followed by a sensitivity analysis to determine which parameter contributes more to the variance of ToF. 

\subsection{ToF Distribution}
The ToF is calculated from electronic structure ($E$, $\lambda$ and $\log_{10}(J^2)$) data, that is, ToF is a function with the input of all electronic structure parameters:
\begin{equation}
    %\tau = g(\lambda, \{E_i|i=1,2,\cdots,N \}, \{J_{ij}|i,j=1,2,\cdots,N \}) \;
    \tau = g(x_1, x_2, \cdots, x_{N_d})
    \label{eq:tau1}
\end{equation}
The input $(x_1, x_2, \cdots, x_{N_d})$ represents the collection of the parameters $$(\lambda, \{E_i|i=1,2,\cdots,N \}, \{J_{ij}|i,j=1,2,\cdots,N \})$$  where $N_d$ is the parameter dimension, that is the number of input parameters. 
And $\{E_i|i=1,2,\cdots,N \}$ is the ordered set whose elements are the molecule energy,
$\{J_{ij}|i,j=1,2,\cdots,N \}$ is the ordered set whose elements are the coupling elements. 

Each parameter has uncertainty caused by the exchange-correlation functional in the multiscale model.
To estimate the uncertainty which is unknown, we consider the maximum amount of uncertainties according to the obtained $(\lambda, \{E_i|i=1,2,\cdots,N \}, \{J_{ij}|i,j=1,2,\cdots,N \})$ with the six different HFX.
To achieve this, each parameter can be modeled as a normal distribution with the mean and standard deviation obtained by the maximum likelihood estimation. For a normal distribution, they are respectively the sample mean and standard deviation obtained by six different HFXs.

Then Monte Carlo scheme is used to estimate the distribution of the ToF. The Monte Carlo scheme has the procedure:
\begin{enumerate}
\item For all $i,j=1,2,\cdots, N_d$, obtain a realization $(x_1, x_2, \cdots, x_{N_d})$ where each parameter has the normal distribution, meaning the corresponding electronic parameters are sampled from the distribution $\lambda_i \in \mathcal{N}(\mathbb{E}(\lambda), \mathbb{V}(\lambda))$, $E_i \in \mathcal{N}(\mathbb{E}(E_i), \mathbb{V}(E_i))$, $\log_{10}(J_{ij}^2) \in \mathcal{N}(\mathbb{E}(\log_{10}(J_{ij}^2)), \mathbb{V}(\log_{10}(J_{ij}^2)))$. 
\item Calculate ToF using the sampled data set $(x_1, x_2, \cdots, x_{Nd})$. 
\item Repeat step 1 and 2 for $N_\text{MC} = 50000$ times to obtain ToFs. Plot the distribution of the ToFs.
\item Estimate the 99\% confidence interval around the median.
\end{enumerate}


To understand how the uncertainty pass from $E, \lambda, J_{ij}$ to ToF, the parameter control is used. That is, a set of parameter, such as $E_i$ for all $i=1,2,\cdots,N$ is choose while keeping other parameter fixed, then performed Monte Carlo sampling to obtain ToF distribution and estimate upper and lower confidence bound.
Then we use Sobol total indices to analyze and quantify 
which uncertainty among $E, \lambda, J_{ij}$ has a greater impact on the ToF, 

%
\begin{figure}
  \centering
  \includegraphics[width=0.45\textwidth]{figs/fig_mle_MADN_withE.pdf}
  \caption{\bjoern{TODO Zhong:}{these should also strictly be $\log_{10}(\tau/\unit[]{s})$.}Distribution of ToFs in the multiscale modeled MADN system.
  Monte Carlo sampling with a sample size $N_\text{MC}=50000$ is used to obtain the sampled distribution. The red vertical lines indicate the lower bound and upper bound of the 99\% confidence interval around the median.
  The black circles indicate the ToF obtained using varying HFX values.
  (a) $P(\log_{10}(\tau)|E_i \text{ uncertain})$, 
  (b) $P(\log_{10}(\tau)|\lambda \text{ uncertain})$, 
  (c) $P(\log_{10}(\tau)|J_{ij} \text{ uncertain})$, 
  (d) $P(\log_{10}(\tau)|E_i, \lambda, J_{ij} \text{ uncertain})$. }
  \label{fig:mle_MADN_withE}
\end{figure}
%


The distribution that we want to obtained during the parameter control are as followed:
\begin{enumerate}
    \item Fixed the $\lambda$ to be $\mathbb{E}(\lambda)$, and $\log_{10}(J_{ij}^2)$ to be $\mathbb{E}(\log_{10}(J_{ij}^2))$. Then each molecule energy $E_i$ is sampled from the normal distribution $\mathcal{N}(\mathbb{E}(E_i), \mathbb{V}(E_i))$. Finally obtain and plot the histogram $N_\text{MC}$ sample of $\tau$, whose distribution is denoted as $P(\log_{10}(\tau)|E_i \text{ uncertain})$.
    \item Fixed all $E_i$ to be $\mathbb{E}(E_i)$ and all $\log_{10}(J_{ij}^2)$ to be $\mathbb{E}(\log_{10}(J_{ij}^2))$. Then $\lambda$ is sampled from the normal distribution $\mathcal{N}(\mathbb{E}(\lambda), \mathbb{V}(\lambda))$. Finally obtain and plot the histogram  $N_\text{MC}$ sample of $\tau$, whose distribution is denoted as $P(\log_{10}(\tau)|\lambda \text{ uncertain})$. 
    \item Fixed the $\lambda$ to be $\mathbb{E}(\lambda)$ and all $E_i$ to be $\mathbb{E}(E_i)$.     Then $J_{ij}$ is sampled from the normal distribution $\log_{10}(J_{ij}^2) \in \mathcal{N}(\mathbb{E}(\log_{10}(J_{ij}^2)), \mathbb{V}(\log_{10}(J_{ij}^2)))$. Finally obtain and plot the histogram  $N_\text{MC}$ sample of $\tau$, whose distribution is denoted as $P(\log_{10}(\tau)|J_{ij} \text{ uncertain})$.
    \item Both $E_i$, $\lambda$ and $J_{ij}$ are sampled from their normal distribution:
    
    $\lambda_i \in \mathcal{N}(\mathbb{E}(\lambda), \mathbb{V}(\lambda))$, 
    
    $E_i \in \mathcal{N}(\mathbb{E}(E), \mathbb{V}(E))$, 
    
    $\log_{10}J_{ij}^2 \in \mathcal{N}(\mathbb{E}(\log_{10}J_{ij}^2), \mathbb{V}(\log_{10}J_{ij}^2))$.
    Then obtain and plot the histogram $N_\text{MC}$ sample of $\tau$, whose distribution is denoted as the following distribution: 
    $P(\log_{10}(\tau)|E_i, \lambda, J_{ij} \text{ uncertain})$.
\end{enumerate}

The ToF distribution for the MADN systems with energy disorder is shown in Fig.\ref{fig:mle_MADN_withE}.
The lower and upper bounds in Fig. \ref{fig:mle_MADN_withE} shows that when only the energies $E_i$ are uncertain, the $\log_{10}(\tau)$ has lower -8.0 and upper bound -6.5. This lower and upper bound is close to that of $P(\log_{10}(\tau)|E_i, \lambda, J_{ij} \text{ uncertain})$, meaning that varying the energy alone generates a $\tau$ distribution close to that when $E_i, \lambda, J_{ij}$ are uncertain. 

Figure .\ref{fig:mle_MADN_withE}(b) shows that $\log_{10}(\tau)$ has a confidence interval of -8.0 to -7.4 when $\lambda$ is varied. 
Figure \ref{fig:mle_MADN_withE}(c) show that when $J_{ij}$ is uncertain $\log_{10}(\tau)$ has a narrow confidence interval of -7.9 to -7.7, suggesting that the variation in $J_{ij}$ leads to relatively small change in $\log_{10}(\tau)$ compared to $E_i$ and $\lambda$.

%
\begin{figure}[t]
  \centering
  \includegraphics[width=0.45\textwidth]{figs/fig_mle_MADN_noE.pdf}
  \caption{\bjoern{TODO Zhong:}{these should also strictly be $\log_{10}(\tau/\unit[]{s})$.}Distribution of ToFs in the multiscale modeled MADN system.
  Monte Carlo sampling with a sample size $N_\text{MC}=50000$ is used to obtain the sampled distribution. The red vertical lines indicate the lower bound and upper bound of the 99\% confidence interval around the median.
  Energy disorder is not consider. 
  The black circles indicate the ToF obtained using varying HFX values.
  (a) $P(\log_{10}(\tau)|\lambda \text{ uncertain})$, 
  (b) $P(\log_{10}(\tau)|J_{ij} \text{ uncertain})$, 
  (c) $P(\log_{10}(\tau)|\lambda, J_{ij} \text{ uncertain})$. }
  \label{fig:mle_MADN_noE}
\end{figure}
%

When the energy disorder in MADN system is not considered, that is, all the MADN molecule energy are set to be equal giving $\Delta E=0$, the distribution of ToF is shown in Fig.\ref{fig:mle_MADN_noE}. 
The distribution $P(\log_{10}(\tau)|\lambda \text{ uncertain})$ has a range of confidence interval -9.6 to -9.0 resembles that of $P(\log_{10}(\tau)|\lambda, J_{ij} \text{ uncertain})$, which has a range of confidence interval -9.7 to -9.1. 

Measuring the sensitivity of each electronic parameter to ToF is to measure each electronic parameter's contribution to the variance of ToF. 
That is, if ToF is most sensitive to one parameter, the variance in this particular parameter will contribute most to the ToF variance. 


One way of decomposing the variance of the model output into fractions attributed to input parameters is the variance-based sensitivity analysis. The local sensitive analysis is to use the partial derivative $\frac{\partial \tau}{\partial x_i}$. While the global sensitivity analysis can use Sobol's indices. Given a model of Equation \ref{eq:tau1}, to measure the parameter $x_i$'s contribution to $\tau$ including all variance caused by its interaction with other parameters $\{x_k, k \neq i \}$, the total effect Sobol's index is calculated as \cite{saltelli_variance_2010}:
\begin{equation}
    S_{T,i} = \frac{ \mathbb{E}_{\mathbf{x}_{\sim i}}[ \mathbb{V}_{x_i}(\tau|\mathbf{x}_{\sim i}) ] }{ \mathbb{V}(\tau) }
    \label{eq:STi}
\end{equation}
The details of the notation is as followed: the vector $\mathbf{x}=(x_1, x_2, \cdots, x_{N_d})$, and $\mathbf{x}_{\sim i}$ denotes the vector of all entries but $x_i$. 
$\mathbb{V}_{x_i}(\tau|\mathbf{x}_{\sim i})$ means the variance of $\tau$ given a set of $\mathbf{x}_{\sim i}$ taken over $x_i$. And $ \mathbb{E}_{\mathbf{x}_{\sim i}}[\cdot]$ denotes the mean of argument $(\cdot)$ taken over all factors but $x_i$.

%
\begin{table}[tbp]%The best place to locate the table environment is directly after its first reference in text
\caption{\label{tab:Sobol}%
Sobol indices of $\lambda$, $E$ and $\log_{10} J^2$ when ToF is calculated with energy disorder, and Sobol indices of $\lambda$ and $\log_{10} J^2$ when ToF is calculated without energy disorder.
}
\begin{ruledtabular}
  \begin{center}
    \begin{tabular}{c c c c c c c} %\hline
      &  \multicolumn{3}{c}{\bf Energy Disorder} & &\multicolumn{2}{c}{\bf no Disorder}\\\cline{2-4}  \cline{6-7}
      parameter  & $\lambda$ & $ E_i$ & $ \log_{10}(J_{ij}^2)$ && $\lambda$ & $ \log_{10}(J_{ij}^2)$ \\ \hline
      %\multicolumn{6}{c}{\bf GRW for $\mathbf{N_c=1}$}\\
   $S_T$  & 0.097 & 0.95 & 0.018 && 0.96 & 0.11 \\
    \end{tabular}
  \end{center}
\end{ruledtabular}
\end{table}
%
Then the $\lambda$ contribution to the total variance is $S_{T,i=1}$.
The energy contribution to the total variance is $S_{T,E} = \sum\limits_{i=1}^{N+1} S_{T,i}$ where $N=1000$ is the number of energy parameter, and the coupling element contribution due to variation in $\log_{10}(J_{ij}^2)$ to the total variance is $S_{T,J}=\sum\limits_{i=N+2}^{N_d} S_{T,i}$. 
Using the quasi Monte Carlo method\cite{sobol_global_2001} with a sample size $N_\text{QMC}=1000$ to calculate $S_{T,i}$, the results are shown in the table \ref{tab:Sobol}.

From table \ref{tab:Sobol} one can see that most of the total variance can be attributed to the energy $E$. In contrast to $E$, the uncertainty in $\lambda$ and $\log_{10} J^2$ contributes much less compared to $E$, this is consistent to the Fig.\ref{fig:mle_MADN_withE}, where the distribution in Fig.\ref{fig:mle_MADN_withE}(d) resembles the distribution in Fig.\ref{fig:mle_MADN_withE}(a).
And Fig.\ref{fig:mle_MADN_withE}(c) shows that the uncertainty in $\log_{10}(J^2)$ only leads to small variances.
When there is no energy disorder, the variance in $\lambda$ has the most contribution to the total variance as shown in Table \ref{tab:Sobol}.

%%%%%%%%%%%%%%%%%%%%%%%%%%%%%%%%%%%%%%%%%%%%%%%%%%%%%%%%%%%%%%%%%%%
%
%\begin{figure}
%  \centering
%  \includegraphics[width=0.45\textwidth]{figs/fig_mle_MADN_withE_SS.pdf}
%  \caption{Distribution of steady state velocity $v_\text{SS}$ in the multiscale modeled MADN.
%  Monte Carlo sampling with a sample size $N_\text{MC}=50000$ is used to obtain the sampled distribution.
%  The red vertical lines indicate the lower bound and upper bound of the 99\% confidence interval around the median.
%  The black circles indicate the $v_\text{SS}$ obtained using varying HFX values.
%  (a) $P(\log_{10}(v_\text{SS})|E_i \text{ uncertain})$, 
%  (b) $P(\log_{10}(v_\text{SS})|\lambda \text{ uncertain})$, 
%  (c) $P(\log_{10}(v_\text{SS})|J_{ij} \text{ uncertain})$, 
%  (d) $P(\log_{10}(v_\text{SS})|E_i, \lambda, J_{ij} \text{ uncertain})$. }
%  \label{fig:mle_MADN_withE_SS}
%\end{figure}
%
%
%\begin{figure}
%  \centering
%  \includegraphics[width=0.45\textwidth]{figs/fig_mle_MADN_noE_SS.pdf}
%  \caption{Distribution of steady state velocity $v_\text{SS}$ in the multiscale modeled MADN. All molecule energies are set to be zero.  
%  Monte Carlo sampling with a sample size $N_\text{MC}=50000$ is used to obtain the sampled distribution.
%  The red vertical lines indicate the lower bound and upper bound of the 99\% confidence interval around the median.
%  The black circles indicate the $v_\text{SS}$ obtained using varying HFX values.
%  (a) $P(\log_{10}(v_\text{SS})|\lambda \text{ uncertain})$, 
%  (b) $P(\log_{10}(v_\text{SS})|J_{ij} \text{ uncertain})$, 
%  (c) $P(\log_{10}(v_\text{SS})|\lambda, J_{ij} \text{ uncertain})$.  }
%  \label{fig:mle_MADN_noE_SS}
%\end{figure}


%\section{Distribution of Steady State Velocity}

%The distribution of steady state velocity of MADN is shown as Fig.\ref{fig:mle_MADN_withE_SS} and \ref{fig:mle_MADN_noE_SS}. 
%From the confidence interval, $P(\log_{10}(v_\text{SS})|E_i \text{ uncertain})$ has a range of -2.6 to -0.9, and the confidence interval $P(\log_{10}(v_\text{SS})|\lambda \text{ uncertain})$ has a range from -1.7 to -1.1. The width of those confidence intervals is very close to the confidence interval of distributions $P(\log_{10}(\tau)|E_i \text{ uncertain})$ and $P(\log_{10}(\tau)|\lambda \text{ uncertain})$, respectively.
%However, $P(\log_{10}(v_\text{SS})|J_{ij} \text{ uncertain})$ has a range from -1.6 to 1.3, spanning 3 orders of magnitude. This is  much larger than $P(\log_{10}(\tau)|J_{ij} \text{ uncertain})$ as shown in Fig .\ref{fig:mle_MADN_withE}.  
%The large confidence interval in $P(\log_{10}(v_\text{SS})|J_{ij} \text{ uncertain})$ shows that $v_{ss}$ is very sensitive to the change in $J_{ij}$. Combining all the uncertainty effects in $E_i, \lambda, J_{ij}$, the confidence interval of $P(\log_{10}(v_\text{SS})|E_i, \lambda, J_{ij} \text{ uncertain})$ is from -2.5 to 1.3, covering 4 orders of magnitude.  

%When energy disorder is not considered, the $v_\text{SS}$ distribution $P(\log_{10}(v_\text{SS})|\lambda, J_{ij} \text{ uncertain})$ has a confidence interval from 0.3 to 3.5, spanning 3 orders of magnitude. This interval is close to the range of $P(\log_{10}(v_\text{SS})|J_{ij} \text{ uncertain})$ when only $J_{ij}$ is uncertain. 
%In contrast, the interval of $P(\log_{10}(v_\text{SS})|\lambda \text{ uncertain})$ is from 0.6 to 1.2, whose width is much smaller than that of $P(\log_{10}(v_\text{SS})|J_{ij} \text{ uncertain})$.

%These result suggests that compared to the ToF $\tau$, the quantity $v_\text{SS}$ is significantly affected by uncertainties in $J_{ij}$, so $v_\text{SS}$ is less robust when the parameters $E_i, \lambda, J_{ij}$ contains uncertainty at the same time. 
%In the next section, the Poole–Frenkel behavior, that is the dependence of charge mobility on the electric field will be study using the ToF setting. 
%%%%%%%%%%%%%%%%%%%%%%%%%%%%%%%%%%%%%%%%%%%%%%%%%%%%%%%%%%%%%%%%%%%%%%%%%%

\section{Charge mobility and PF Behavior}

In this section, we want to study the dependence of drift mobility on the HFX. The charge mobility is calculated as:
\begin{equation}
    \mu = \frac{\vec{v} \vec{F} }{ |\vec{F}|^2}
    \label{eq:mu}
\end{equation}
To consider the charge dynamics along X, Y and Z axis in the ToF setting, the ToF is calculated with different settings of \textit{Source}-\textit{Sink} and $\vec{F}$ combinations.

\begin{itemize}
    \item Setting the molecules with $r^x < \min(r^x)+0.5$ as \textit{Source}, and the molecules with $r^x > \max(r^x)-0.5$ as \textit{Sink}, and $\vec{F}^T = (F,0,0)$, denote the calculated the ToF as $\tau_\text{Xmin}$, and the mobility is as $\mu_\text{Xmin} = \frac{L_x }{\tau_\text{Xmin} |\vec{F}|}$.  
    \item Setting the molecules with $r^x > \max(r^x)-0.5$ as \textit{Source}, and the molecules with $r^x < \min(r^x)+0.5$ as \textit{Sink}, and $\vec{F}^T = (-F,0,0)$, denote the calculated the ToF as $\tau_\text{Xmax}$, and the mobility is as $\mu_\text{Xmax} = \frac{L_x }{\tau_\text{Xmax} |\vec{F}|}$.  
    \item Setting the molecules with $r^y < \min(r^y)+0.5$ as \textit{Source}, and the molecules with $r^y > \max(r^y)-0.5$ as \textit{Sink}, and $\vec{F}^T = (0,F,0)$, denote the calculated the ToF as $\tau_\text{Ymin}$, and the mobility is as $\mu_\text{Ymin} = \frac{L_y }{\tau_\text{Ymin} |\vec{F}|}$. 
    \item Setting the molecules with $r^y > \max(r^y)-0.5$ as \textit{Source}, and the molecules with $r^y > \max(r^y)-0.5$ as \textit{Sink}, and $\vec{F}^T = (0,-F,0)$, denote the calculated the ToF as $\tau_\text{Ymax}$, and the mobility is as $\mu_\text{Ymax} = \frac{L_y }{\tau_\text{Ymax} |\vec{F}|}$. 
    \item Setting the molecules with $r^z < \min(r^z)+0.5$ as \textit{Source}, and the molecules with $r^z > \max(r^z)-0.5$ as \textit{Sink}, and $\vec{F}^T = (0,0,F)$, denote the calculated the ToF as $\tau_\text{Zmin}$, and the mobility is as $\mu_\text{Zmin} = \frac{L_z }{\tau_\text{Zmin} |\vec{F}|}$. 
    \item Setting the molecules with $r^z > \max(r^z)-0.5$ as \textit{Source}, and the molecules with $r^z > \min(r^z)+0.5$ as \textit{Sink}, and $\vec{F}^T = (0,0,-F)$, denote the calculated the ToF as $\tau_\text{Zmax}$, and the mobility is as $\mu_\text{Zmax} = \frac{L_z }{\tau_\text{Zmax} |\vec{F}|}$. 
\end{itemize}
Finally the ToF drift mobility is calculated as:
\begin{equation}
    \mu_\text{PF} = \frac{1}{6} (\mu_\text{Xmin}+\mu_\text{Xmax}+\mu_\text{Ymin}+\mu_\text{Ymax}+\mu_\text{Zmin}+\mu_\text{Zmax})
\end{equation}

When the $\lambda, E_i, J_{ij}$ are uncertain and sampled from the normal distribution, the distributions of charge mobility obtained from MC sample under specific electric field are shown in Fig.\ref{fig:fig_mle_withE_mu2_ave}.
%
\begin{figure}
    \centering
    \includegraphics[width=0.45\textwidth]{figs/fig_mle_withE_mu2_ave.pdf}
    \caption{\bjoern{TODO Zhong:}{these should also strictly be $\log_{10}(\mu/(\unit[]{cm^2(Vs)^{-1}}))$; there should be no PF label at $\mu$.} Distribution of $\mu_\text{PF}$ in the multiscale modeled MADN system.
    Monte Carlo sampling with a sample size $N_\text{MC}=10000$ is used to obtain the sampled distribution. The red vertical lines indicate the lower bound and upper bound of the 99\% confidence interval around the median. (a) $|\vec{F}|=4 \times 10^7 \unit{V/m}$, (b) $|\vec{F}|=5 \times 10^7 \unit{V/m}$, (c) $|\vec{F}|=6 \times 10^7 \unit{V/m}$, (d) $|\vec{F}|=7 \times 10^7 \unit{V/m}$, (e) $|\vec{F}|=8 \times 10^7 \unit{V/m}$, (f) $|\vec{F}|=9 \times 10^7 \unit{V/m}$.}
    \label{fig:fig_mle_withE_mu2_ave}
\end{figure}
%

When the $\lambda, E_i, J_{ij}$ are uncertain and sampled from the normal distribution, the distributions of charge mobility obtained from MC sample under specific electric field are shown in Fig. \ref{fig:fig_mle_withE_mu2_ave}. The ToF depends on the electric field through the Marcus rate. As the electric increase from $|\vec{F}|=4 \times 10^7 \unit{V/m}$
$|\vec{F}|=9 \times 10^7 \unit{V/m}$, the
range of the 90\% confidence interval where $\log_{10} \mu_\text{PF}$ lies
estimated from the MC sample decrease from 0.90 to 0.69.
So the electric field results in a narrower range of $\mu_\text{PF}$ distribution.
Due to this reason, the $\mu_\text{PF}$ calculated from $\alpha=0$ is not in the 99\% confidence interval around the median of the MC sampled data, although the zero-field ToF of $\alpha=0$ is contained in the according 99\% confidence interval, as shown in Fig. \ref{fig:mle_MADN_withE}(d).

The ToF depends on the electric field through the Marcus rate. 
Poole and Frenkel \cite{frenkel_prebreakdown_1938} predicted the electric-field dependence of charge mobility as $\mu_\text{PF}(F)=\mu_0 \exp (\beta \sqrt{|\vec{F}|})$.
So it is common to plot the mobility $\mu_\text{PF}$ against $\sqrt{|\vec{F}|}$ in a so-called Poole-Frenkel plot.
which is shown in Fig. \ref{fig:PF_plot_ave} for specifie HFX values. The 99\% confidence interval obtained via MC sampled in shown in the green shadow. 
This interval suggests that if the uncertainties from the parameters are represented by the maximum likelihood distribution, the Poole–Frenkel plot from $\alpha=0$ has less 1\% chance to happen and thus is very likely. 
%
\begin{figure}[tbp]
    \centering
    \includegraphics[width=0.45\textwidth]{figs/fig_PF_plot_ave.pdf}
    \caption{\bjoern{TODO Zhong:}{remove PF label; avoid the large white space at the right; make data plot symbols a little bigger and lines a little thicker.}Electric-field dependence of the mobility $\mu$ in the multiscale modeled MADN system. The dash lines are Poole–Frenkel plots obtained with specific HFX value, and the green shadow indicates the 99\% confidence interval estimated from the MC sampling with a sample size of $N_\text{MC}=10000$.}
    \label{fig:PF_plot_ave}
\end{figure}

%
\begin{table}[tbp]%The best place to locate the table environment is directly after its first reference in text
  \caption{\label{tab:PF_parameter}%
  Poole-Frenkel parameters $\mu_0$ (in $\unit[]{cm^2/(Vs)}$) and $\beta$ (in $\unit[]{\sqrt{cm/V}}$) of the multiscale modeled MADN, calculated by the six different HFX values. 
  }
  \begin{ruledtabular}
    \begin{tabular}{c c c c c}
    $\alpha$ & & $\mu_0$  & & $\beta$ \\
      \hline
    0 & & $4.0 \times 10^{-3}$ & & $1.0 \times 10^{-3}$ \\
    0.05 & & $2.1 \times 10^{-3}$ & & $9.3 \times 10^{-4}$ \\
    0.10 & & $1.0 \times 10^{-3}$ & & $1.5 \times 10^{-3}$ \\
    0.15 & & $8.0 \times 10^{-4}$ & & $1.6 \times 10^{-3}$ \\
    0.20 & & $9.3 \times 10^{-4}$ & & $1.3 \times 10^{-3}$ \\
    0.20 & & $2.6 \times 10^{-4}$ & & $1.8 \times 10^{-3}$ \\
      \end{tabular}
  \end{ruledtabular}
  \end{table}
%  
The extracted Poole–Frenkel paratermeter calculated from the six $\alpha$ values are summarized in Table \ref{tab:PF_parameter}. 
Excluding the data of $\alpha=0$, the extracted zero-field mobility $\mu_0$ has a range from $2.6 \times 10^{-4}$ to $2.1 \times 10^{-3}$, and the field effect constant $\beta$ range from $9.3 \times 10^{-4}$ to $1.8 \times 10^{-3}$.
So $\beta$ is less sensitive to the multiscale model parameter $\alpha$ compared the zero-field mobility.
%%%%%%%%%%%%%%%%%%%%%%%%%%%%%%%%%%%%%%%%%%%%%%%%%%%%%%%%%%%%%%%%%%%%
\section{Conclusion}
The uncertainty effect from the exchange-correlation functional of the DFT on the multiscale model of OSC is studied. By choosing different HFX, the calculated values of some electronic structures remain very similar,
such as the reorganization energy $\lambda$, molecular energy $E_i$ and the majority of the coupling elements  $J_{ij}$. While a few small-value coupling elements $J_{ij}$ show large different in calculated values when different HFX is used. Those uncertainties propagate to the output of the multiscale model, resulting in a range of QoI which has a maximum about 15 times of its minimum. These findings high-
light the importance of selecting appropriate functionals to ensure model robustness. Also indicated is that for multiscale model charge mobility, when the values have difference is approximately 10\%, we should interpret the results with caution and refrain from concluding that they represent fundamentally different phenomena.

Assuming the maximum amount of uncertainty from the calculated electronic structure parameters using the six HFX, the 99\% confidence level of the ToFs around the median are estimated. The Sobol indexes show that the molecular energy has the most contribution to the total variance, followed by the reorganization energy and coupling elements. Our results also shows that due to the uncertainties, the steady state velocity distribution spans 4 orders of magnitude. So it is less robust when
the electronic structure parameters are uncertain, and a useful range of velocity is hard to obtain. Future research should explore the impact of other sources of uncertainty and extend this framework to different types of semiconducting materials, thus enhancing the predictive power and applicability of multiscale models in the field.


\bibliography{references}

\appendix*
\section{Electrostatic and polarized energy contributions}
\begin{figure*}
  \centering
  \includegraphics[width=0.80\textwidth]{figs/scatterEstat_qmmm.pdf}
  \caption{\bjoern{TODO Zhong:}{add labels (a)...; replace $\alpha$ with $\alpha_\text{HF}$.}Scatter plot of MADN electrostatic energy calculated from different HFX, compared to the electrostatic energy calculated from HFX=0.25 (The PBE0 functional). The brighter color near the diagonal lines indicates denser population of the molecules.  The top and right histogram show the energy distributions.}
  \label{fig:Estat_qmmm_MADN}
\end{figure*}

\begin{figure*}
  \centering
  \includegraphics[width=0.80\textwidth]{figs/scatterEdip_qmmm.pdf}
  \caption{\bjoern{TODO Zhong:}{add labels (a)...; replace $\alpha$ with $\alpha_\text{HF}$.}Scatter plot of MADN polarization energy calculated from different HFX, compared to the polarization energy calculated from HFX=0.25 (The PBE0 functional). The brighter color near the diagonal lines indicates denser population of the molecules.  The top and right histogram show the energy distributions.}
  \label{fig:Edip_qmmm_MADN}
\end{figure*}


\end{document}
